\documentclass[a4paper]{article}

\def\npart {IV}
\def\nterm {Spring}
\def\nyear {2021}
\def\nlecturer {梁灿彬}
\def\ncourse {微分几何}

\RequirePackage{etex}                           %导入包的一种方式,与usepackage类似
\makeatletter                                   %实现公式编号与节号关联
\ifx \nauthor\undefined                         %如果没有定义nauthor
  \def\nauthor{Liu Ming}                        %定义nauthor
\else                                           %否则
\fi                                             %结束if语句

\author{Based on lectures by \nlecturer \\      %定义作者信息
  \small Notes taken by \nauthor}               %定义作者信息
\date{\nterm\ \nyear}                           %定义日期

\usepackage{alltt}                              %定义一个alltt环境,类似verbatim抄录环境,只是环境中的"\、{、}"这三个字符仍具有latex作用意义
\usepackage{amsfonts}                           %定义了大写空心粗体字命令\mathbb和欧拉字体命令\mathfrak以及数学公式中各种相应的字体,如数学斜体和粗希腊字母下标、求和积分等大符号下标、欧拉数学字体、斯拉夫字体等
\usepackage{amsmath}                            %数学类宏包,用来改进和提高方程式、多行上、下标等数学结构的排版效果
\usepackage{amssymb}                            %定义了amsfonts宏包里msam和mabm字库中全部数学符号的命令
\usepackage{amsthm}                             %定义了一个proof环境,用来排版定理和证明,能自动在最后添加证毕符号,还提供\newtheorem{定理环境名}[计数器名]{标题}来自定义定理类环境
\usepackage{booktabs}                           %画三线表,线条精细可变
\usepackage{caption}                            %提供浮动环境(图形和表格)中自定义caption的方法
\usepackage{enumitem}                           %提供对三种基本列表环境的布局控制:枚举、逐项列出和描述
\usepackage{fancyhdr}                           %用于页眉页脚的设置
\usepackage{graphicx}                           %该包建立在graphics包的基础上,为\includegraphics命令可选参数键值接口
\usepackage{mathdots}                           %重新定义了\ddots和\vdots,并定义了\iddots
\usepackage{mathtools}                          %基于amsmath提供更多数学功能的包
\usepackage{microtype}                          %调整全篇文章的字间距,也可以调整某种或某几种字号的字间距,不支持中文
\usepackage{multirow}                           %提供了 \multirow 命令,可以在表格中排版横跨两行以上的文本
\usepackage{pdflscape}                          %提供改变页面方向的库,对于pdflscape环境内的页面将横置
\usepackage{pgfplots}                           %Pgfplots是一种可视化工具,可简化在文档中包含绘图的过程。基本思想是,用户提供输入数据/公式,然后pgfplots 宏包会帮助用户绘制响应的图像
\usepackage{siunitx}                            %提供物理单位、物理量的输入
\usepackage{textcomp}                           %提供生成特殊符号的宏包,包括泰铢、项目符号、版权、音符、小节和日元等
\usepackage{slashed}                            %提供字符上画斜线的包,例如\slashed{D}
\usepackage{tabularx}                           %根据表格的总宽度自动计算特定的表列宽度
\usepackage{tikz}                               %绘图包
\usepackage{tkz-euclide}                        %绘制平面图形包
\usepackage[normalem]{ulem}                     %提供各种线的包(下划线、波浪线、割线、删除线)
\usepackage[all]{xy}                            %绘制交换图宏包
\usepackage{imakeidx}                           %自动处理索引项
\usepackage{ctex}                               %提供了一个统一的中文 LaTeX 文档框架

\makeindex[intoc, title=Index]
\indexsetup{othercode={\lhead{\emph{Index}}}}

\ifx \nextra \undefined
  \usepackage[hidelinks,
    pdfauthor={Liu Ming},
    pdfsubject={Learning Notes: Part \npart\ - \ncourse},
    pdftitle={Part \npart\ - \ncourse},
    pdfkeywords={Learning \npart\ \nterm\ \nyear\ \ncourse}]{hyperref}
  \title{Part \npart\ --- \ncourse}
\else
  \usepackage[hidelinks,
    pdfauthor={Liu Ming},
    pdfsubject={Learning Notes: Part \npart\ - \ncourse\ (\nextra)},
    pdftitle={Part \npart\ - \ncourse\ (\nextra)},
    pdfkeywords={Learning \npart\ \nterm\ \nyear\ \ncourse\ \nextra}]{hyperref}

  \title{Part \npart\ --- \ncourse \\ {\Large \nextra}}
  \renewcommand\printindex{}
\fi

\pgfplotsset{compat=1.12}

\pagestyle{fancyplain}
\ifx \ncoursehead \undefined
  \def\ncoursehead{\ncourse}
\fi

\lhead{\emph{\nouppercase{\leftmark}}}
\ifx \nextra \undefined
  \rhead{
    \ifnum\thepage=1
    \else
      \npart\ \ncoursehead
    \fi}
\else
  \rhead{
    \ifnum\thepage=1
    \else
      \npart\ \ncoursehead \ (\nextra)
    \fi}
\fi
\usetikzlibrary{arrows.meta}
\usetikzlibrary{decorations.markings}
\usetikzlibrary{decorations.pathmorphing}
\usetikzlibrary{positioning}
\usetikzlibrary{fadings}
\usetikzlibrary{intersections}
\usetikzlibrary{cd}

\newcommand*{\Cdot}{{\raisebox{-0.25ex}{\scalebox{1.5}{$\cdot$}}}}
\newcommand {\pd}[2][ ]{
  \ifx #1 { }
    \frac{\partial}{\partial #2}
  \else
    \frac{\partial^{#1}}{\partial #2^{#1}}
  \fi
}
\ifx \nhtml \undefined
\else
  \renewcommand\printindex{}
  \DisableLigatures[f]{family = *}
  \let\Contentsline\contentsline
  \renewcommand\contentsline[3]{\Contentsline{#1}{#2}{}}
  \renewcommand{\@dotsep}{10000}
  \newlength\currentparindent
  \setlength\currentparindent\parindent

  \newcommand\@minipagerestore{\setlength{\parindent}{\currentparindent}}
  \usepackage[active,tightpage,pdftex]{preview}
  \renewcommand{\PreviewBorder}{0.1cm}

  \newenvironment{stretchpage}%
  {\begin{preview}\begin{minipage}{\hsize}}%
        {\end{minipage}\end{preview}}
  \AtBeginDocument{\begin{stretchpage}}
      \AtEndDocument{\end{stretchpage}}

  \newcommand{\@@newpage}{\end{stretchpage}\begin{stretchpage}}

  \let\@real@section\section
  \renewcommand{\section}{\@@newpage\@real@section}
  \let\@real@subsection\subsection
  \renewcommand{\subsection}{\@ifstar{\@real@subsection*}{\@@newpage\@real@subsection}}
\fi
\ifx \ntrim \undefined
\else
  \usepackage{geometry}
  \geometry{
    papersize={379pt, 699pt},
    textwidth=345pt,
    textheight=596pt,
    left=17pt,
    top=54pt,
    right=17pt
  }
\fi

\ifx \nisofficial \undefined
  \let\@real@maketitle\maketitle
  \renewcommand{\maketitle}{\@real@maketitle\begin{center}\begin{minipage}[c]{0.9\textwidth}\centering\footnotesize These notes are not endorsed by the lecturers, and I have modified them (often significantly) after lectures. They are nowhere near accurate representations of what was actually lectured, and in particular, all errors are almost surely mine.\end{minipage}\end{center}}
\else
\fi

% Theorems
\theoremstyle{definition}
\newtheorem*{aim}{Aim}
\newtheorem*{axiom}{Axiom}
\newtheorem*{claim}{Claim}
\newtheorem*{cor}{Corollary}
\newtheorem*{conjecture}{Conjecture}
\newtheorem*{defi}{定义}
\newtheorem*{eg}{Example}
\newtheorem*{ex}{Exercise}
\newtheorem*{fact}{Fact}
\newtheorem*{law}{Law}
\newtheorem*{lemma}{Lemma}
\newtheorem*{notation}{Notation}
\newtheorem*{prop}{Proposition}
\newtheorem*{question}{Question}
\newtheorem*{problem}{Problem}
\newtheorem*{rrule}{Rule}
\newtheorem*{thm}{定理}
\newtheorem*{assumption}{Assumption}
\newtheorem*{concept}{基本概念}

\newtheorem*{remark}{Remark}
\newtheorem*{warning}{Warning}
\newtheorem*{exercise}{Exercise}

\newtheorem{ndefi}{定义}[subsection]
\newtheorem{nthm}{定理}[subsection]
\newtheorem{nlemma}[nthm]{Lemma}
\newtheorem{nprop}[nthm]{Proposition}
\newtheorem{ncor}[nthm]{Corollary}


\renewcommand{\labelitemi}{--}
\renewcommand{\labelitemii}{$\circ$}
\renewcommand{\labelenumi}{(\roman{*})}

\let\stdsection\section
\renewcommand\section{\newpage\stdsection}

% Strike through
\def\st{\bgroup \ULdepth=-.55ex \ULset}


%%%%%%%%%%%%%%%%%%%%%%%%%
%%%%% Maths Symbols %%%%%
%%%%%%%%%%%%%%%%%%%%%%%%%

\let\U\relax
\let\C\relax
\let\G\relax

% Matrix groups
\newcommand{\GL}{\mathrm{GL}}
\newcommand{\Or}{\mathrm{O}}
\newcommand{\PGL}{\mathrm{PGL}}
\newcommand{\PSL}{\mathrm{PSL}}
\newcommand{\PSO}{\mathrm{PSO}}
\newcommand{\PSU}{\mathrm{PSU}}
\newcommand{\SL}{\mathrm{SL}}
\newcommand{\SO}{\mathrm{SO}}
\newcommand{\Spin}{\mathrm{Spin}}
\newcommand{\Sp}{\mathrm{Sp}}
\newcommand{\SU}{\mathrm{SU}}
\newcommand{\U}{\mathrm{U}}
\newcommand{\Mat}{\mathrm{Mat}}

% Matrix algebras
\newcommand{\gl}{\mathfrak{gl}}
\newcommand{\ort}{\mathfrak{o}}
\newcommand{\so}{\mathfrak{so}}
\newcommand{\su}{\mathfrak{su}}
\newcommand{\uu}{\mathfrak{u}}
\renewcommand{\sl}{\mathfrak{sl}}

% Special sets
\newcommand{\C}{\mathbb{C}}
\newcommand{\CP}{\mathbb{CP}}
\newcommand{\GG}{\mathbb{G}}
\newcommand{\N}{\mathbb{N}}
\newcommand{\Q}{\mathbb{Q}}
\newcommand{\R}{\mathbb{R}}
\newcommand{\RP}{\mathbb{RP}}
\newcommand{\T}{\mathbb{T}}
\newcommand{\Z}{\mathbb{Z}}
\renewcommand{\H}{\mathbb{H}}

% Brackets
\newcommand{\abs}[1]{\left\lvert #1\right\rvert}
\newcommand{\bket}[1]{\left\lvert #1\right\rangle}
\newcommand{\brak}[1]{\left\langle #1 \right\rvert}
\newcommand{\braket}[2]{\left\langle #1\middle\vert #2 \right\rangle}
\newcommand{\bra}{\langle}
\newcommand{\ket}{\rangle}
\newcommand{\norm}[1]{\left\lVert #1\right\rVert}
\newcommand{\normalorder}[1]{\mathop{:}\nolimits\!#1\!\mathop{:}\nolimits}
\newcommand{\tv}[1]{|#1|}
\renewcommand{\vec}[1]{\boldsymbol{\mathbf{#1}}}

% not-math
\newcommand{\bolds}[1]{{\bfseries #1}}
\newcommand{\cat}[1]{\mathsf{#1}}
\newcommand{\ph}{\,\cdot\,}
\newcommand{\term}[1]{\emph{#1}\index{#1}}
\newcommand{\phantomeq}{\hphantom{{}={}}}
% Probability
\DeclareMathOperator{\Bernoulli}{Bernoulli}
\DeclareMathOperator{\betaD}{beta}
\DeclareMathOperator{\bias}{bias}
\DeclareMathOperator{\binomial}{binomial}
\DeclareMathOperator{\corr}{corr}
\DeclareMathOperator{\cov}{cov}
\DeclareMathOperator{\gammaD}{gamma}
\DeclareMathOperator{\mse}{mse}
\DeclareMathOperator{\multinomial}{multinomial}
\DeclareMathOperator{\Poisson}{Poisson}
\DeclareMathOperator{\var}{var}
\newcommand{\E}{\mathbb{E}}
\newcommand{\Prob}{\mathbb{P}}

% Algebra
\DeclareMathOperator{\adj}{adj}
\DeclareMathOperator{\Ann}{Ann}
\DeclareMathOperator{\Aut}{Aut}
\DeclareMathOperator{\Char}{char}
\DeclareMathOperator{\disc}{disc}
\DeclareMathOperator{\dom}{dom}
\DeclareMathOperator{\fix}{fix}
\DeclareMathOperator{\Hom}{Hom}
\DeclareMathOperator{\id}{id}
\DeclareMathOperator{\image}{image}
\DeclareMathOperator{\im}{im}
\DeclareMathOperator{\re}{re}
\DeclareMathOperator{\tr}{tr}
\DeclareMathOperator{\Tr}{Tr}
\newcommand{\Bilin}{\mathrm{Bilin}}
\newcommand{\Frob}{\mathrm{Frob}}

% Others
\newcommand\ad{\mathrm{ad}}
\newcommand\Art{\mathrm{Art}}
\newcommand{\B}{\mathcal{B}}
\newcommand{\cU}{\mathcal{U}}
\newcommand{\Der}{\mathrm{Der}}
\newcommand{\D}{\mathrm{D}}
\newcommand{\dR}{\mathrm{dR}}
\newcommand{\exterior}{\mathchoice{{\textstyle\bigwedge}}{{\bigwedge}}{{\textstyle\wedge}}{{\scriptstyle\wedge}}}
\newcommand{\F}{\mathbb{F}}
\newcommand{\G}{\mathcal{G}}
\newcommand{\Gr}{\mathrm{Gr}}
\newcommand{\haut}{\mathrm{ht}}
\newcommand{\Hol}{\mathrm{Hol}}
\newcommand{\hol}{\mathfrak{hol}}
\newcommand{\Id}{\mathrm{Id}}
\newcommand{\lie}[1]{\mathfrak{#1}}
\newcommand{\op}{\mathrm{op}}
\newcommand{\Oc}{\mathcal{O}}
\newcommand{\pr}{\mathrm{pr}}
\newcommand{\Ps}{\mathcal{P}}
\newcommand{\pt}{\mathrm{pt}}
\newcommand{\qeq}{\mathrel{``{=}"}}
\newcommand{\Rs}{\mathcal{R}}
\newcommand{\Vect}{\mathrm{Vect}}
\newcommand{\wsto}{\stackrel{\mathrm{w}^*}{\to}}
\newcommand{\wt}{\mathrm{wt}}
\newcommand{\wto}{\stackrel{\mathrm{w}}{\to}}
\renewcommand{\d}{\mathrm{d}}
\renewcommand{\P}{\mathbb{P}}
%\renewcommand{\F}{\mathcal{F}}


\let\Im\relax
\let\Re\relax

\DeclareMathOperator{\area}{area}
\DeclareMathOperator{\card}{card}
\DeclareMathOperator{\ccl}{ccl}
\DeclareMathOperator{\ch}{ch}
\DeclareMathOperator{\cl}{cl}
\DeclareMathOperator{\cls}{\overline{\mathrm{span}}}
\DeclareMathOperator{\coker}{coker}
\DeclareMathOperator{\conv}{conv}
\DeclareMathOperator{\cosec}{cosec}
\DeclareMathOperator{\cosech}{cosech}
\DeclareMathOperator{\covol}{covol}
\DeclareMathOperator{\diag}{diag}
\DeclareMathOperator{\diam}{diam}
\DeclareMathOperator{\Diff}{Diff}
\DeclareMathOperator{\End}{End}
\DeclareMathOperator{\energy}{energy}
\DeclareMathOperator{\erfc}{erfc}
\DeclareMathOperator{\erf}{erf}
\DeclareMathOperator*{\esssup}{ess\,sup}
\DeclareMathOperator{\ev}{ev}
\DeclareMathOperator{\Ext}{Ext}
\DeclareMathOperator{\fst}{fst}
\DeclareMathOperator{\Fit}{Fit}
\DeclareMathOperator{\Frac}{Frac}
\DeclareMathOperator{\Gal}{Gal}
\DeclareMathOperator{\gr}{gr}
\DeclareMathOperator{\hcf}{hcf}
\DeclareMathOperator{\Im}{Im}
\DeclareMathOperator{\Ind}{Ind}
\DeclareMathOperator{\Int}{Int}
\DeclareMathOperator{\Isom}{Isom}
\DeclareMathOperator{\lcm}{lcm}
\DeclareMathOperator{\length}{length}
\DeclareMathOperator{\Lie}{Lie}
\DeclareMathOperator{\like}{like}
\DeclareMathOperator{\Lk}{Lk}
\DeclareMathOperator{\Maps}{Maps}
\DeclareMathOperator{\orb}{orb}
\DeclareMathOperator{\ord}{ord}
\DeclareMathOperator{\otp}{otp}
\DeclareMathOperator{\poly}{poly}
\DeclareMathOperator{\rank}{rank}
\DeclareMathOperator{\rel}{rel}
\DeclareMathOperator{\Rad}{Rad}
\DeclareMathOperator{\Re}{Re}
\DeclareMathOperator*{\res}{res}
\DeclareMathOperator{\Res}{Res}
\DeclareMathOperator{\Ric}{Ric}
\DeclareMathOperator{\rk}{rk}
\DeclareMathOperator{\Rees}{Rees}
\DeclareMathOperator{\Root}{Root}
\DeclareMathOperator{\sech}{sech}
\DeclareMathOperator{\sgn}{sgn}
\DeclareMathOperator{\snd}{snd}
\DeclareMathOperator{\Spec}{Spec}
\DeclareMathOperator{\spn}{span}
\DeclareMathOperator{\stab}{stab}
\DeclareMathOperator{\St}{St}
\DeclareMathOperator{\supp}{supp}
\DeclareMathOperator{\Syl}{Syl}
\DeclareMathOperator{\Sym}{Sym}
\DeclareMathOperator{\vol}{vol}

\pgfarrowsdeclarecombine{twolatex'}{twolatex'}{latex'}{latex'}{latex'}{latex'}
\tikzset{->/.style = {decoration={markings,
    mark=at position 1 with {\arrow[scale=2]{latex'}}},
postaction={decorate}}}
\tikzset{<-/.style = {decoration={markings,
    mark=at position 0 with {\arrowreversed[scale=2]{latex'}}},
postaction={decorate}}}
\tikzset{<->/.style = {decoration={markings,
    mark=at position 0 with {\arrowreversed[scale=2]{latex'}},
    mark=at position 1 with {\arrow[scale=2]{latex'}}},
postaction={decorate}}}
\tikzset{->-/.style = {decoration={markings,
    mark=at position #1 with {\arrow[scale=2]{latex'}}},
postaction={decorate}}}
\tikzset{-<-/.style = {decoration={markings,
    mark=at position #1 with {\arrowreversed[scale=2]{latex'}}},
postaction={decorate}}}
\tikzset{->>/.style = {decoration={markings,
    mark=at position 1 with {\arrow[scale=2]{latex'}}},
postaction={decorate}}}
\tikzset{<<-/.style = {decoration={markings,
    mark=at position 0 with {\arrowreversed[scale=2]{twolatex'}}},
postaction={decorate}}}
\tikzset{<<->>/.style = {decoration={markings,
    mark=at position 0 with {\arrowreversed[scale=2]{twolatex'}},
    mark=at position 1 with {\arrow[scale=2]{twolatex'}}},
postaction={decorate}}}
\tikzset{->>-/.style = {decoration={markings,
    mark=at position #1 with {\arrow[scale=2]{twolatex'}}},
postaction={decorate}}}
\tikzset{-<<-/.style = {decoration={markings,
    mark=at position #1 with {\arrowreversed[scale=2]{twolatex'}}},
postaction={decorate}}}

\tikzset{circ/.style = {fill, circle, inner sep = 0, minimum size = 3}}
\tikzset{scirc/.style = {fill, circle, inner sep = 0, minimum size = 1.5}}
\tikzset{mstate/.style={circle, draw, blue, text=black, minimum width=0.7cm}}

\tikzset{eqpic/.style={baseline={([yshift=-.5ex]current bounding box.center)}}}
\tikzset{commutative diagrams/.cd,cdmap/.style={/tikz/column 1/.append style={anchor=base east},/tikz/column 2/.append style={anchor=base west},row sep=tiny}}

\definecolor{mblue}{rgb}{0.2, 0.3, 0.8}
\definecolor{morange}{rgb}{1, 0.5, 0}
\definecolor{mgreen}{rgb}{0.1, 0.4, 0.2}
\definecolor{mred}{rgb}{0.5, 0, 0}

\def\drawcirculararc(#1,#2)(#3,#4)(#5,#6){%
  \pgfmathsetmacro\cA{(#1*#1+#2*#2-#3*#3-#4*#4)/2}%
  \pgfmathsetmacro\cB{(#1*#1+#2*#2-#5*#5-#6*#6)/2}%
  \pgfmathsetmacro\cy{(\cB*(#1-#3)-\cA*(#1-#5))/%
    ((#2-#6)*(#1-#3)-(#2-#4)*(#1-#5))}%
  \pgfmathsetmacro\cx{(\cA-\cy*(#2-#4))/(#1-#3)}%
  \pgfmathsetmacro\cr{sqrt((#1-\cx)*(#1-\cx)+(#2-\cy)*(#2-\cy))}%
  \pgfmathsetmacro\cA{atan2(#2-\cy,#1-\cx)}%
  \pgfmathsetmacro\cB{atan2(#6-\cy,#5-\cx)}%
  \pgfmathparse{\cB<\cA}%
  \ifnum\pgfmathresult=1
    \pgfmathsetmacro\cB{\cB+360}%
  \fi
  \draw (#1,#2) arc (\cA:\cB:\cr);%
}
\newcommand\getCoord[3]{\newdimen{#1}\newdimen{#2}\pgfextractx{#1}{\pgfpointanchor{#3}{center}}\pgfextracty{#2}{\pgfpointanchor{#3}{center}}}

\newcommand\qedshift{\vspace{-17pt}}
\newcommand\fakeqed{\pushQED{\qed}\qedhere}

\def\Xint#1{\mathchoice
  {\XXint\displaystyle\textstyle{#1}}%
  {\XXint\textstyle\scriptstyle{#1}}%
  {\XXint\scriptstyle\scriptscriptstyle{#1}}%
  {\XXint\scriptscriptstyle\scriptscriptstyle{#1}}%
  \!\int}
\def\XXint#1#2#3{{\setbox0=\hbox{$#1{#2#3}{\int}$}
      \vcenter{\hbox{$#2#3$}}\kern-.5\wd0}}
\def\ddashint{\Xint=}
\def\dashint{\Xint-}

\newcommand\separator{{\centering\rule{2cm}{0.2pt}\vspace{2pt}\par}}

\newenvironment{own}{\color{gray!70!black}}{}

\newcommand\makecenter[1]{\raisebox{-0.5\height}{#1}}

\mathchardef\mdash="2D

\newenvironment{significant}{\begin{center}\begin{minipage}{0.9\textwidth}\centering\em}{\end{minipage}\end{center}}
\DeclareRobustCommand{\rvdots}{%
  \vbox{
    \baselineskip4\p@\lineskiplimit\z@
    \kern-\p@
    \hbox{.}\hbox{.}\hbox{.}
  }}
\DeclareRobustCommand\tph[3]{{\texorpdfstring{#1}{#2}}}
\makeatother


\begin{document}
\maketitle
\vspace*{10cm}
\begin{center}
    \large{参考教材:} 微分几何和广义相对论[第二版]
\end{center}
\thispagestyle{empty}

\newpage
\tableofcontents
\thispagestyle{empty}

\newpage
\setcounter{page}{1}

\section{拓扑空间介绍}
\subsection{集论初步}

\begin{concept}
    确切地指定了的若干事物的全体叫一个集合,简称集。
\end{concept}
\begin{concept}
    集中的每一事物叫一个元素或点。
\end{concept}

\begin{concept}
    不含元素的集叫空集,记作$ \varnothing  $。
\end{concept}

\begin{ndefi}
    若集$ A $的每一元素都属于集$ X $,就说$ A $是$ X $的\textbf{子集(subset)},也说$ A $含于(is contained in)$ X $或$ X $含(contains)$ A $,记作$ A \subset X $且$ A \neq X $。规定$ varnothing $是任一集合的子集。若$ A \subset X $且$ A \neq X $,则将$ A $称为$ X $的\textbf{真子集(proper subset)}。若$ X \subset Y $且$ Y \subset X $,则称集$ X $和集$ Y $为相等的,记作$ X = Y $。
\end{ndefi}

\begin{ndefi}
    集合$ A,B $的并集、交集、差集和补集定义为: \\
    \textbf{并集(union) $ A \cup B $}$ :=\{x | x \in A \text{或} x \in B\} $; \\
    \textbf{交集(intersection) $ A \cap B $} $ := \{x | x \in A, x \in B \} $; \\
    \textbf{差集(difference) $ A - B $} $ :=\{ x | x \in A, x \notin B \} $; \\
    \textbf{补集(complement) $ -A $} $ := \{ x | x \in X, x \notin A, A \subset X \} $。
\end{ndefi}

\begin{nthm}
    以上集运算符合以下规律: \\
    \textbf{交换律} $ A \cup B = B \cup A, \qquad A \cap B = B \cap A $ \\
    \textbf{结合律} $ (A \cup B) \cup C = A \cup (B \cup C), \quad (A \cap B) \cap C = A \cap (B \cap C) $ \\
    \textbf{分配律} $ (A \cap B) \cup C = (A \cup C) \cap (B \cup C), (A \cup B) \cap C = (A \cap C) \cup (B \cap C) $  \\
    \textbf{De Morgan律} $ A - (B \cup C) = (A - B) \cap (A - C), \qquad A - (B \cap C) = (A - B) \cup (A - C) $
\end{nthm}

\begin{ndefi}
    非空集合$ X,Y $的\textbf{卡氏积(Cartesian product) $ X \times Y $}定义为
    \begin{equation*}
        X \times Y := \{ (x,y) | x \in X , y \in Y \}
    \end{equation*}
    就是说,$ X \times Y $是一个由有序对$ (x,y) $组成的集合,其中每一元素是由$ X $的一个元素$ x $和$ Y $的一个元素$ y $组成。多个(有限个)集合的卡氏积可类似地定义为:
    \begin{equation*}
        X \times Y \times Z := \{ (x,y,z) | x \in X, y \in Y, z \in Z \}
    \end{equation*}
    卡氏积满足结合律,即$ (X \times Y) \times Z = X \times (Y \times Z) $。
\end{ndefi}

\begin{ndefi}
    $ \R^{n} $的任意两个元素$ x=(x^{1}, \cdots, x^{n}), y=(y^{1}, \cdots, y^{n}) $之间的\textbf{距离(distance)}$ \abs{y-x} $定义为$ \abs{y-x}:=\sqrt{\sum_{i=1}^{n}(y^{i}-x^{i})^{2}} $
\end{ndefi}

\begin{ndefi}
    设$ X,Y $为非空集合,一个从$ X $到$ Y $的\textbf{映射(map)},记作$ f: X \to Y $,是一个法则,它给$ X $的每一元素指定$ Y $的唯一的对应元素。若$ y \in Y $是$ x \in X $的对应元素,就写$ y=f(x) $,并称$ y $为$ x $在映射$ f $下的\textbf{像(image)},称$ x $为$ y $的\textbf{原像(或逆像inverse image)}。$ X $称为映射$ f $的\textbf{定义域(domain)},$ X $的全体元素在映射$ f $下的像的集合(记作$ f[X] $)称为映射$ f: X \to Y $的\textbf{值域(range)}。若$ f(x)=f^{'}(x) \quad \forall x \in X $。
    通常把$ y=f(x) $写成$ f:x \mapsto y $。
\end{ndefi}

\begin{ndefi}
    若任一$ y \in Y $有不多于一个逆像(可以没有),则映射$ f: X \to Y $叫\textbf{一一的(onr-to-one)}。若任一$ y \in Y $都有逆像(可多于一个),则映射$ f:X \to Y $叫\textbf{到上的(onto)}。
\end{ndefi}

\begin{ndefi}
    若$ f(x)=f(x^{'}) \quad \forall x,x^{'} \in X $,则$ f:X \to Y $称为常值映射。
\end{ndefi}

\begin{ndefi}
    设$ X,Y,Z $为集,$ f: X \to Y $和$ g:Y \to Z $为映射,则$ f $和$ g $的\textbf{复合映射$ f \circ g $}是从$ X $到$ Z $的映射,定义为$ (f \circ g)(x):=g(f(x)) \in Z \quad \forall x \in X $。
\end{ndefi}


\subsection{拓扑空间}

\begin{concept}
    $ \R $的子集分为开子集和非开子集两大类,开子集具有三个性质,对任意非空集合$ X $,可指定其中某些子集是开的,其他为非开的。开子集的三大性质分别为:
    \begin{enumerate}
        \item $ X $本身和空集$ \varnothing $为开子集;
        \item 有限个开子集之交为开子集;
        \item 任意个(可以有限个也可以无限个)开子集之并为开子集。
    \end{enumerate}
\end{concept}

\begin{concept}
    拓扑结构为每种满足上述三要求的指定给集合$ X $赋予的附加结构,定义了拓扑结构的集合$ X $的全体开子集也组成一个集合,称为$ X $的一个\textbf{拓扑(topology)},记作$ \mathcal{T}  $。
\end{concept}

\begin{ndefi}
    非空集合$ X $的一个\textbf{拓扑(topology)}$ \mathcal{T} $是$ X $的若干子集的集合,满足:
    \begin{enumerate}
        \item $ X, \varnothing \in \mathcal{T} $;
        \item 若$ O_{i} \in \mathcal{T}, i=1,2,\cdots,n $,则$ \bigcap_{i=1}^{n} O_{i} \in \mathcal{T} $;
        \item 若$ O_{a} \in \mathcal{T} \forall a $,则$ \bigcup_{a} O_{a} \in \mathcal{T} $。
    \end{enumerate}
\end{ndefi}

\begin{ndefi}
    指定了拓扑$ \mathcal{T} $的集合$ X $称为\textbf{拓扑空间(topological space)}。若$ O \in \mathcal{T} $,则拓扑空间$ X $的子集$ O $称为\textbf{开子集}(简称\textbf{开集})。
\end{ndefi}

\begin{concept}
    设$ X $为任意非空集合,令$ \mathcal{T} $为$ X $的全部子集的集合,则构成$ X $的一个拓扑,叫\textbf{离散拓扑(discrete topology)}。
\end{concept}

\begin{concept}
    设$ X $为任意非空集合,令$ \mathcal{T} = \{ X, \varnothing \} $,则构成$ X $的一个拓扑,叫\textbf{凝聚拓扑(indiscrete topology)}。
\end{concept}

凝聚拓扑是元素最少的拓扑,离散拓扑是元素最多的拓扑。

\begin{concept}
    设$ X $为任意非空集合,令$ \mathcal{T} := \{ \text{空集或}\R $中能表为开区间之并的子集$ \} $ 称为$ \R $的\textbf{通常拓扑}。
\end{concept}

\begin{concept}
    设$ X = \R^{n} $,则$ \mathcal{T} := \{\text{空集或}\R $中能表为开球之并的子集$ \} $称为$ \R^{n} $的\textbf{通常拓扑},其中\textbf{开球(open ball)}定义为$ B(x_{0}, r) := \{ x \in \R^{n} | \abs{x - x_0} < r \} $,$ x_{0} $称为球心,$ r > 0 $称为半径。
\end{concept}

\begin{concept}
    设$ (X_{1}, \mathcal{T}_{1}) $和$ (X_{2}, \mathcal{T}_{2}) $为拓扑空间,$ X = X_{1} \times X_{2} $,定义$ X $的拓扑为:
    \begin{equation}
        \mathcal{T} := \{ O \subset X | O \text{可表为形如}O_{1} \times O_{2} \text{的子集之并,} O_{1} \in \mathcal{T}_{1}, O_{2} \in \mathcal{T}_{2} \}
    \end{equation}
    则$ \mathcal{T} $称为$ X $的\textbf{乘积拓扑(product topology)}。
\end{concept}

\begin{concept}
    设$ (X, \mathcal{T}) $是拓扑空间,$ A $为$ X $的任一非空子集,把$ A $看成集合,当然也可指定拓扑(记作$ \mathcal{S} $)是$ A $成为拓扑空间,记作$ (A, \mathcal{S}) $。由于$ A $是$ X $的子集,那么定义
    \begin{equation*}
        \mathcal{S} := \{ V \subset A | \exists O \in \mathcal{T} \text{使} V = A \cap O \}
    \end{equation*}
    定义这样的$ \mathcal{S} $叫做$ A(\subset X) $的、由$ \mathcal{T} $导出的\textbf{诱导拓扑(induced topology)}。$ (A, \mathcal{S}) $称为$ ( X, \mathcal{T} ) $的\textbf{拓扑子空间(topological subspace)}。
\end{concept}

\begin{ndefi}
    设$ (X, \mathcal{T}) $和$ (Y, \mathcal{S}) $为拓扑空间。若$ f^{-1}[O] \in \mathcal{T} \quad \forall O \in \mathcal{S} $,则映射$ f:X \to Y $称为\textbf{连续的(continuous)}。
\end{ndefi}

\begin{ndefi}
    设$ (X, \mathcal{T}) $和$ (Y, \mathcal{S}) $为拓扑空间。若$ \forall $满足$ f(x) \in G^{'} $的$ G^{'} \in \mathcal{S}, \exists G \in \mathcal{T} $使$ x \in G $且$ f[G] \subset G^{'} $,则映射$ f: X \to Y $称为\textbf{在点$ x \in X $处连续}。若它在所有点$ x \in X $上连续,则$ f: X \to Y $称为\textbf{连续}。
\end{ndefi}

\begin{ndefi}
    若$ \exists $映射$ f: X \to Y $,满足:
    \begin{enumerate}
        \item $ f $是一一到上的;
        \item $ f $及$ f^{-1} $都连续
    \end{enumerate}
    则拓扑空间$ (X, \mathcal{T}) $和$ (Y, \mathcal{S}) $称为\textbf{相互同胚(homeomorphic to each other)}。这样的$ f $称为从$ (X, \mathcal{T}) $到$ (Y, \mathcal{S}) $的\textbf{同胚映射},简称\textbf{ 同胚(homeomorphic) }。
\end{ndefi}

\begin{concept}
    普通函数$ y=f(x) $的连续性和可微性用$ C^{r} $表示,其中$ r $为非负整数,$ C^{0} $代表连续,$ C^{r} $代表$ r $阶导函数存在并连续,$ C^{\infty} $代表任意阶导函数存在并连续[称为\textbf{光滑(smooth)}]。
\end{concept}

\begin{ndefi}
    若$ \exists O \in \mathcal{T} $使$ x \in O \subset N $,则$ N \subset X $称为$ x \in X $的一个\textbf{领域(neighborhood)}。自身是开基的领域称为\textbf{开领域}。
\end{ndefi}

\begin{ndefi}(子集的领域)
    若$ \exists O \in \mathcal{T} $使$ A \subset O \subset N $,则$ N \subset X $称为$ A \subset X $的一个\textbf{领域}。
\end{ndefi}

\begin{nthm}
    $ A \subset X $是开集,当且仅当$ A $是$ x $的领域$ \forall x \in A $。
\end{nthm}

\begin{ndefi}
    若$ -C \in \mathcal{T} $,则$ C \subset X $叫\textbf{闭集(closed set)}。
\end{ndefi}

\begin{nthm}
    闭集有以下性质:
    \begin{enumerate}
        \item 任意个闭集的交集是闭集;
        \item 有限个闭集的并集是闭集;
        \item $ X $及$ \varnothing $是闭集。
    \end{enumerate}
\end{nthm}

\begin{ndefi}
    若除$ X $和$ \varnothing $外没有即开又闭的子集,则拓扑空间$ (X, \mathcal{T}) $称为\textbf{连通的(connected)}。
\end{ndefi}

\begin{ndefi}
    设$ (X, \mathcal{T}) $为拓扑空间,$ A \subset X $。$ A $的\textbf{闭包(closure)}$ \overline{A} $是所有含$ A $的闭集的交集,即
    \begin{equation*}
        \overline{A} := \bigcap_{a} C_{a}. \quad A \subset C_{a}, \quad \text{且} C_{a} \text{为闭。}
    \end{equation*}
\end{ndefi}

\begin{ndefi}
    设$ (X, \mathcal{T}) $为拓扑空间,$ A \subset X $。$ A $的\textbf{内部(interior)}$ i(A) $是所有含于$ A $的开集的并集,即
    \begin{equation*}
        i(A) := \bigcup_{a} O_{a}, \quad O_{a} \subset A, \quad O_{a} \in \mathcal{T}
    \end{equation*}
\end{ndefi}

\begin{ndefi}
    设$ (X, \mathcal{T}) $为拓扑空间,$ A \subset X $。$ A $的\textbf{边界(boundary)}$ \dot{A} := \overline{A} - i(A) $,$ x \in \dot{A} $称为\textbf{边界点}。$ \dot{A} $也记作$ \partial A $。
\end{ndefi}

\begin{nthm}
    $ \overline{A}, i(A), \dot{A} $有以下性质:
    \begin{enumerate}
        \item \begin{itemize}
                  \item $ \overline{A} $为闭集;
                  \item $ A \subset \overline{A} $;
                  \item $ A = \overline{A} $当且仅当$ A $为闭集;
              \end{itemize}
        \item \begin{itemize}
                  \item $ i(A) $为开集;
                  \item $ i(A) \subset A $;
                  \item $ i(A) = A $当且仅当$ A \in \mathcal{T} $;
              \end{itemize}
        \item $ \dot{A} $为闭集。
    \end{enumerate}
\end{nthm}

\begin{ndefi}
    若$ A \subset \bigcup_{a} O_{a} $,则$ X $的开子集的集合$ \{ O_{a} \} $叫$ A \subset X $的一个\textbf{开覆盖(open cover)}。也可以说$ \{ O_{a} \} $覆盖$ A $。
\end{ndefi}

\subsection{紧致性}

\begin{ndefi}
    设$ \{ O_{a} \} $是$ A \subset X $的开覆盖(open cover),若$ \{ O_{a} \} $的有限个元素构成的子集$ \{ O_{a_{1}}, \cdots, O_{a_{n}} \} $也覆盖$ A $,就说$ \{ O_{a} \} $有\textbf{有限子覆盖(finite subcover)}。
\end{ndefi}

\end{document}